\chapter{Аналитическая часть}

\section{Рекурсия}

Рекурсия --- функция, которая вызывает сама себя.~\cite{recursion_def}  
Хвостовая рекурсия --- рекурсия, при которой рекурсивный вызов является последней операцией перед возвратом из функции.~\cite{tail_recursion}

\section{Граф}

Пусть \( V \) --- непустое множество, \( V^{(2)} \) --- множество всех его двухэлементных подмножеств. Пара \( (V, E) \), где \( E \) --- произвольное подмножество множества \( V^{(2)} \), называется \textbf{графом} (неориентированным графом). Элементы множества \( V \) называются \textbf{вершинами} графа, а элементы множества \( E \) --- \textbf{рёбрами}. Итак, граф --- это конечное множество \( V \) вершин и множество \( E \) рёбер, \( E \subseteq V^{(2)} \)~\cite{graph_def}.

Если направление рёбер не указано, то граф называется \textbf{неориентированным}. Если направление рёбер указано, то граф называется \textbf{ориентированным}, а сами рёбра принято называть \textbf{дугами}~\cite{graph_def}.

\section{Виды графов}

\subsection{Граф управления}

\textbf{Граф управления} --- это ориентированный граф, вершины которого соответствуют базовым блокам программы, а дуги показывают возможные переходы управления между ними.  
Граф управления используется для анализа структуры программы, оптимизации и построения пути выполнения алгоритма.

\subsection{Информационный граф}

\textbf{Информационный граф} --- это граф, в котором вершины представляют данные, а рёбра отражают отношения или зависимости между ними.  
Такой граф описывает потоки данных в системе и помогает анализировать, как информация перемещается и преобразуется между различными компонентами программы.

\subsection{Граф операционной истории}

\textbf{Граф операционной истории} --- это граф, отображающий последовательность выполнения операций программы.  
Его вершины соответствуют операциям или командам, а рёбра отражают причинно-следственные связи между ними.  
Такой граф позволяет анализировать порядок выполнения действий и выявлять зависимости между операциями.

\subsection{Граф информационной истории}

\textbf{Граф информационной истории} --- это граф, описывающий эволюцию данных в процессе выполнения программы.  
Его вершины представляют состояния данных после каждой операции, а рёбра отражают, как одно состояние информации переходит в другое.  
Граф информационной истории используется для отслеживания изменений данных и анализа корректности преобразований.


\section*{Вывод}

В аналитической части были рассмотрены основные понятия, необходимые для анализа алгоритмов.  
Были даны определения рекурсии и хвостовой рекурсии, а также определён неориентированный и ориентированный граф с пояснением структуры вершин и рёбер.  

Кроме того, рассмотрены виды графов, применяемые при анализе программ:  
граф управления для построения потоков выполнения, информационный граф для анализа передачи данных, граф операционной истории для отслеживания последовательности выполнения операций и граф информационной истории для анализа эволюции данных в процессе работы программы.  

Данное теоретическое основание позволяет проводить систематический анализ алгоритмов и оценивать структуру и поведение программ.

