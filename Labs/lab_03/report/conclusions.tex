\chapter*{ЗАКЛЮЧЕНИЕ}

В ходе выполнения работы были рассмотрены графовые модели рекурсивного и итеративного алгоритмов вывода элементов последовательности с нечётными номерами, а также проведён их детальный анализ. Поставленная цель --- на основе построенных графовых моделей выявить участки программного кода, допускающие параллельное исполнение, --- была полностью достигнута. В результате анализа установлено, что такие участки отсутствуют.

В результате работы:
\begin{itemize}
	\item разработаны и реализованы два варианта алгоритма --- рекурсивный и итеративный --- в соответствии с заданием;
	\item для каждого алгоритма построены четыре типа графовых моделей: граф управления, информационный граф, граф операционной истории и граф информационной истории;
	\item проведён структурный анализ всех моделей с целью выявления возможностей параллелизма;
	\item установлено, что оба алгоритма обладают строго последовательной структурой выполнения: каждая операция зависит от результата предыдущей, а рекурсивный вызов формирует линейную цепочку активаций без ветвлений или независимых веток;
	\item подтверждено, что ни одна из рассмотренных реализаций не содержит фрагментов, пригодных для распараллеливания.
\end{itemize}

Таким образом, несмотря на различия в организации кода (использование цикла в итеративном варианте и рекурсивных вызовов в рекурсивном), оба подхода принципиально последовательны и не допускают одновременного выполнения операций. Это накладывает естественное ограничение на применение многопоточности или векторизации при решении подобных задач. Полученные выводы согласуются с теоретическими свойствами линейных алгоритмов обработки последовательностей и подтверждают корректность построенных графовых моделей.