\ssr{ЗАКЛЮЧЕНИЕ}

Цель работы достигнута: разработаны и реализованы алгоритмы умножения матриц, а также проведён их сравнительный анализ с учётом процессорного времени выполнения.

Все задачи выполнены:
\begin{enumerate}
	\item описаны математические основы стандартного алгоритма умножения матриц и алгоритма Винограда;
	\item предоставлено подробное описание алгоритмов в виде схем, а также предложена оптимизация алгоритма Винограда;
	\item выполнена теоретическая оценка вычислительной трудоёмкости трёх алгоритмов (стандартного, Винограда и оптимизированного Винограда). Согласно оценке, оптимизированный алгоритм Винограда должен быть наименее трудоёмким, а стандартный алгоритм и неоптимизированный Виноград --- более затратными;
	\item реализованы и протестированы все три алгоритма умножения матриц;
	\item проведены измерения процессорного времени работы реализаций для различных размеров квадратных матриц;
	\item выполнен сравнительный анализ на основе полученных данных. Практические замеры подтвердили, что оптимизированный алгоритм Винограда является наиболее быстрым для матриц размером больше 4 элементов, ускоряя классический алгоритм Винограда в среднем на 23\% и стандартный алгоритм на 44\%. Для очень маленьких матриц (1–4 элемента) стандартный алгоритм работает быстрее оптимизированного Винограда, в среднем на 14\%.
\end{enumerate}

Таким образом, проведённая работа позволила подтвердить эффективность оптимизации алгоритма Винограда и дать количественную оценку ускорения по сравнению с другими алгоритмами умножения матриц.
