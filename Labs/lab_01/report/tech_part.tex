\chapter{Технологическая часть}

В данной части приведён выбор инструментов для разработки и измерения времени работы программ, представлены листинги реализованных алгоритмов умножения, а также результаты функционального тестирования.

\section{Средства реализации}

Для разработки алгоритмов и программного обеспечения использовался язык программирования \textit{C++}~\cite{cpp}. 
Этот язык обладает статической типизацией, что соответствует требованиям, предъявляемым к лабораторным работам по курсу анализа алгоритмов.

Для измерения процессорного времени применялись встроенные функции из заголовочного файла \textit{x86intrin.h}~\cite{intrin}, 
который предоставляет доступ к низкоуровневым инструкциям процессора в языке \textit{C++}.

\section{Реализации алгоритмов}

В листинге~\ref{lst_std_algo} приведена реализация стандартного алгоритма умножения матриц.

\begin{lstlisting}[label=lst_std_algo, caption=Реализация стандартного алгоритма умножения матриц]
	int simple_multiplication(size_t n, size_t m, size_t t,
	double **matrix_A, double **matrix_B, double **result_matrix)
	{
		if (n != t)
		return ERROR;
		
		for (size_t i = 0; i < n; i++)
		{
			for (size_t j = 0; j < t; j++)
			{
				for (size_t k = 0; k < m; k++)
				{
					result_matrix[i][j] += matrix_A[i][k] * matrix_B[k][j];
				}
			}
		}
		return OK;
	}
\end{lstlisting}

Реализации алгоритмов Винограда и его оптимизированной версии вынесены в Приложения (Листинги~\ref{lst_vino_algo}--\ref{lst_vino_opt_algo}).

\section{Тестирование реализаций алгоритмов}

В таблице~\ref{table_tests} приведены примеры функциональных испытаний реализованных алгоритмов умножения матриц.  
\begin{table}[H]
	\caption{Функциональные тесты}
	\label{table_tests}
	\begin{center}
		\resizebox{\textwidth}{!}{\begin{tabular}{|x{1cm}|x{6cm}|x{3.5cm}|x{2.5cm}|x{4cm}|}
				\hline
				\multirow{2}{*}{№}
				&\multirow{2}{*}{Описание теста}
				&\multicolumn{2}{c|}{Входные данные}
				&\multirow{2}{*}{Ожидаемый результат}
				\\ \cline{3-4}
				
				&
				&\multicolumn{1}{c|}{Матрица 1}
				& Матрица 2
				&
				\\ \hline
				
				1
				& Попытка задать матрицу с отрицательным числом столбцов
				& (Количество столбцов = -2)
				& $\begin{pmatrix}
					1 & 0\\
					0 & 1
				\end{pmatrix}$
				& Сообщение об ошибке ввода
				\\ \hline
				
				2
				& Перемножение матриц несогласованных размеров (2x3 и 4x2)
				& $\begin{pmatrix}
					1 & 2 & 3\\
					4 & 5 & 6
				\end{pmatrix}$
				& $\begin{pmatrix}
					1 & 0\\
					0 & 1\\
					1 & 0\\
					0 & 1
				\end{pmatrix}$
				& Ошибка: операция невозможна
				\\ \hline
				
				3
				& Корректное умножение квадратных матриц 2x2
				& $\begin{pmatrix}
					1 & 2\\
					3 & 4
				\end{pmatrix}$
				& $\begin{pmatrix}
					2 & 0\\
					1 & 2
				\end{pmatrix}$
				& $\begin{pmatrix}
					4 & 4\\
					10 & 8
				\end{pmatrix}$
				\\ \hline
				
				4
				& Перемножение прямоугольных матриц (2x3 и 3x2)
				& $\begin{pmatrix}
					1 & 0 & -1\\
					2 & 3 & 4
				\end{pmatrix}$
				& $\begin{pmatrix}
					1 & 2\\
					0 & 1\\
					-1 & 0
				\end{pmatrix}$
				& $\begin{pmatrix}
					2 & 2\\
					-2 & 8
				\end{pmatrix}$
				\\ \hline
				
				5
				& Умножение единичной матрицы на произвольную
				& $\begin{pmatrix}
					1 & 0 & 0\\
					0 & 1 & 0\\
					0 & 0 & 1
				\end{pmatrix}$
				& $\begin{pmatrix}
					5 & -1 & 2\\
					0 & 3 & 7\\
					1 & 1 & 1
				\end{pmatrix}$
				& $\begin{pmatrix}
					5 & -1 & 2\\
					0 & 3 & 7\\
					1 & 1 & 1
				\end{pmatrix}$
				\\ \hline
				
				6
				& Умножение матрицы на нулевую
				& $\begin{pmatrix}
					2 & 3\\
					-1 & 4
				\end{pmatrix}$
				& $\begin{pmatrix}
					0 & 0\\
					0 & 0
				\end{pmatrix}$
				& $\begin{pmatrix}
					0 & 0\\
					0 & 0
				\end{pmatrix}$
				\\ \hline
				
				7
				& Проверка работы с вещественными числами
				& $\begin{pmatrix}
					1.5 & -2.0\\
					0.0 & 3.2
				\end{pmatrix}$
				& $\begin{pmatrix}
					2.0 & 1.0\\
					-1.0 & 0.5
				\end{pmatrix}$
				& $\begin{pmatrix}
					5.5 & 0.5\\
					-3.2 & 1.6
				\end{pmatrix}$
				\\ \hline
				
				8
				& Перемножение матрицы и её транспонированной версии
				& $\begin{pmatrix}
					1 & 2 & 3
				\end{pmatrix}$
				& $\begin{pmatrix}
					1\\
					2\\
					3
				\end{pmatrix}$
				& $\begin{pmatrix}
					14
				\end{pmatrix}$
				\\ \hline
				
				9
				& Умножение на диагональную матрицу
				& $\begin{pmatrix}
					1 & 2\\
					3 & 4
				\end{pmatrix}$
				& $\begin{pmatrix}
					5 & 0\\
					0 & 2
				\end{pmatrix}$
				& $\begin{pmatrix}
					5 & 4\\
					15 & 8
				\end{pmatrix}$
				\\ \hline
				
				10
				& Умножение разреженных матриц (с большим числом нулей)
				& $\begin{pmatrix}
					0 & 0 & 1\\
					0 & 2 & 0\\
					3 & 0 & 0
				\end{pmatrix}$
				& $\begin{pmatrix}
					0 & 4 & 0\\
					0 & 0 & 5\\
					6 & 0 & 0
				\end{pmatrix}$
				& $\begin{pmatrix}
					6 & 0 & 0\\
					0 & 0 & 10\\
					0 & 12 & 0
				\end{pmatrix}$
				\\ \hline
				
				11
				& Проверка симметричных матриц
				& $\begin{pmatrix}
					1 & 2\\
					2 & 1
				\end{pmatrix}$
				& $\begin{pmatrix}
					1 & 2\\
					2 & 1
				\end{pmatrix}$
				& $\begin{pmatrix}
					5 & 4\\
					4 & 5
				\end{pmatrix}$
				\\ \hline
				
				12
				& Умножение прямоугольных матриц (3x2 и 2x4)
				& $\begin{pmatrix}
					1 & 0\\
					0 & 1\\
					2 & 3
				\end{pmatrix}$
				& $\begin{pmatrix}
					1 & 2 & 0 & 1\\
					0 & 1 & 1 & 0
				\end{pmatrix}$
				& $\begin{pmatrix}
					1 & 2 & 0 & 1\\
					0 & 1 & 1 & 0\\
					2 & 7 & 3 & 2
				\end{pmatrix}$
				\\ \hline
				
				13
				& Умножение векторов-строк (1x3 на 3x1) с отрицательными числами
				& $\begin{pmatrix} -2 & 4 & 1 \end{pmatrix}$
				& $\begin{pmatrix} 3 \\ -1 \\ 2 \end{pmatrix}$
				& $\begin{pmatrix} -8 \end{pmatrix}$ \\ \hline
				
				14
				& Умножение матрицы с разной размерностью (2x4 на 4x3)
				& $\begin{pmatrix} 1 & 0 & 2 & -1\\ 3 & -1 & 0 & 2 \end{pmatrix}$
				& $\begin{pmatrix} 0 & 1 & 2\\ 3 & -1 & 0\\ 1 & 2 & 1\\ -1 & 0 & 1 \end{pmatrix}$
				& $\begin{pmatrix} 0 & 5 & 5\\ -2 & 2 & 4 \end{pmatrix}$ \\ \hline
				
				15
				& Умножение матрицы на единичную нестандартного размера (3x3 на 3x3)
				& $\begin{pmatrix} 7 & 2 & -1\\ 0 & 3 & 5\\ 4 & -2 & 6 \end{pmatrix}$
				& $\begin{pmatrix} 1 & 0 & 0\\ 0 & 1 & 0\\ 0 & 0 & 1 \end{pmatrix}$
				& $\begin{pmatrix} 7 & 2 & -1\\ 0 & 3 & 5\\ 4 & -2 & 6 \end{pmatrix}$ \\ \hline
				
				
				
				
		\end{tabular}}
	\end{center}
\end{table}

Все приведённые тесты успешно пройдены, что подтверждает правильность работы алгоритмов в различных ситуациях.  

\section*{Вывод}

В технологической части были рассмотрены средства реализации и способ измерения процессорного времени. Реализованы и приведены листинги трёх алгоритмов умножения матриц: классического, алгоритма Винограда и его оптимизированной версии. С помощью расширенного набора функциональных тестов подтверждена корректность работы всех реализаций.  

\clearpage


