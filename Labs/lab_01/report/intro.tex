\ssr{ВВЕДЕНИЕ}

В рамках данной лабораторной работы анализируются алгоритмы умножения матриц: стандартный и метод Винограда, который рассматривается в базовой и оптимизированной модификациях.
Целью исследования является сравнительный анализ производительности и особенностей реализации указанных алгоритмов.
Для достижения цели были поставлены следующие задачи:

\begin{enumerate}
	\item изложить математические принципы, лежащие в основе классического метода умножения матриц и алгоритма Винограда.
	\item представить схемы, описывающие логику работы трёх рассматриваемых алгоритмов: стандартного, базового Винограда и его оптимизированной версии.
	\item сформулировать вычислительную модель и провести в её рамках теоретический анализ трудоёмкости трёх изучаемых алгоритмов умножения матриц.
	\item разработать программные реализации трёх алгоритмов умножения.
	\item провести исследование по измерению процессорного времени, затрачиваемого выполнением реализованных алгоритмов.
	\item выполнить сравнительную характеристику алгоритмов.
\end{enumerate}
\newpage

\clearpage
