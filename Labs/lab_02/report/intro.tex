\ssr{ВВЕДЕНИЕ}

Целью данной работы является сравнительный анализ рекурсивного и нерекурсивного алгоритмов. Для достижения поставленной цели необходимо выполнить следующие задачи:

\begin{enumerate}
	\item разработать рекурсивный и нерекурсивный алгоритмы решения задачи, согласно варианту;
	\item описать средства разработки и инструменты замера процессорного времени выполнения реализации алгоритмов;
	\item реализовать разработанные алгоритмы;
	\item выполнить тестирование реализации алгоритмов;
	\item выполнить теоретическую оценку затрачиваемой реализацией каждого алгоритма памяти (для рекурсивного алгоритма --- на материале анализа высоты дерева рекурсивных вызовов и оценки затрачиваемой на один вызов функции памяти);
	\item выполнить замеры процессорного делением времени выполнения к времени выполнения реализации алгоритмов в зависимости от варьируемого входа (одна точка на графике получается идентичных расчетов на k, k > 100);
	\item оценить трудоемкость двух алгоритмов или их реализаций в худшем случае (если случаев несколько);
	\item сравнить результаты замеров процессорного времени и оценки трудоемкости;
	\item сделать выводы из сравнительного анализа реализации рекурсивного и нерекурсивного алгоритмов решения одной и той же задачи, заданной вариантом, по критериям ёмкостной эффективности на материале теоретической оценки пиковой временной эффективности (на материале результатов измерений).

\end{enumerate}
\newpage

\clearpage
