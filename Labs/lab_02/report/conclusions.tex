\chapter*{ЗАКЛЮЧЕНИЕ}

В ходе выполнения работы были рассмотрены алгоритмы вывода элементов последовательности с нечетными номерами и проведено их сравнение. Поставленная цель — сравнительный анализ рекурсивного и итеративного алгоритмов — была достигнута.

В результате работы:
\begin{itemize}
	\item разработаны рекурсивный и итеративный алгоритмы вывода элементов последовательности с нечетными номерами;
	\item описаны средства разработки (язык программирования C++) и инструменты точного замера процессорного времени с использованием функции \texttt{read\_tsc()}, реализованной через инструкцию \texttt{RDTSC};
	\item реализованы и протестированы оба алгоритма для различных размеров входной последовательности;
	\item выполнена теоретическая оценка сложности алгоритмов: оба варианта имеют линейную трудоемкость $O(n)$;
	\item произведены замеры времени работы при увеличении длины входных данных;
	\item проведён анализ ёмкостной эффективности реализаций: рекурсивный алгоритм требует дополнительную память под стек вызовов — $O(n)$, в то время как итеративный использует постоянный объем памяти — $O(1)$;
	\item сравнены практические результаты замеров с теоретическими оценками: полученные данные подтвердили линейный характер роста времени выполнения обоих алгоритмов.
\end{itemize}

Практические исследования подтвердили теоретические выводы: несмотря на одинаковую асимптотическую сложность $O(n)$, итеративный алгоритм продемонстрировал лучшие временные показатели и более эффективное использование памяти за счёт отсутствия накладных расходов на рекурсивные вызовы. 

Таким образом, при решении задач подобного типа предпочтительно применять итеративный подход, обеспечивающий более высокую производительность и стабильность работы программы, а использование \texttt{read\_tsc()} позволило получить максимально точные результаты замеров процессорного времени.
