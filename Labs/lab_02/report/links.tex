\addcontentsline{toc}{chapter}{СПИСОК ИСПОЛЬЗОВАННЫХ ИСТОЧНИКОВ}
\begin{thebibliography}{}
	\bibitem{recursion_def} Tproger. Что такое рекурсия и как с ней работать [Электронный ресурс]. Режим доступа: \url{https://tproger.ru/articles/chto-takoe-rekursiya-i-kak-s-nej-rabotat} (Дата обращения: 12.10.2025).
	\bibitem{tail_recursion} ScalaBook. Хвостовая рекурсия в функциональном программировании [Электронный ресурс]. Режим доступа: \url{https://scalabook.ru/fp/fp/tail_recursion.html} (Дата обращения: 12.10.2025).
	\bibitem{cpp} Metanit.com. Руководство по C++ [Электронный ресурс]. Режим доступа: \url{https://metanit.com/cpp/tutorial/} (Дата обращения: 12.10.2025).
	\bibitem{intrin} GCC documentation: x86 Intrinsics <x86intrin.h> [Электронный ресурс]. Режим доступа: \url{https://gcc.gnu.org/onlinedocs/gcc/x86-Built-in-Functions.html} (Дата обращения: 12.10.2025).
\end{thebibliography}

